\NeedsTeXFormat{LaTeX2e}[2005/12/01]
%%    2009/03/12 v1.0 GAUBM Vorlage fuer Abschlussarbeiten Physik
%% Template fuer Bachelor- und Masterarbeiten
%% an der Fakultaet fuer Physik (c) Thomas Pruschke der GA Universitaet
%% Verbesserungsvorschlaege bitte an pruschke@theorie.physik.uni-goettingen.de
%%
%% Benoetigte Pakete: datenumber
%%

%%%%%%%%%%%%%%%%%%%%%%%%%%%%%%%%%%%%%%%%%%%%%%%%%%%%%%%%%%%%%%%%%%%%%%
%%%%%%%%%% Bitte vor dem Veraendern diese Datei umbenennen! %%%%%%%%%%
%%%%%%%%%%%%%%%%%%%%%%%%%%%%%%%%%%%%%%%%%%%%%%%%%%%%%%%%%%%%%%%%%%%%%%

%% scrbook - Ersatz fuerr LaTeX book Klasse aus dem KOMA Script
%% Moegliche Optionen: diejenigen der Klasse scrbook ausser titlepage
%% Updates, Fixes und Modifikationen von Boris Lemmer, 07.01.2015

%% deutsche Arbeit:
\documentclass[master,       %% Typ der Arbeit: bachelor oder master
               twoside,        %% zweiseitiges Layout
               BCOR10mm,       %% Bindekorrektur 10 mm
%               liststotoc,nomtotoc,bibtotoc, %% Aufnahme der div. Verzeichnisse
                                              %% ins Inhaltsverzeichnis
%               english,ngerman, %% Alternativspr. Englisch, Dokumentspr. Deutsch
               ngerman,english  %% Alternativspr. Deutsch, Dokumentspr. Englisch
%               final,          %% Endversion; draft fuer schnelles Kompilieren
               ]{GAUBM}

\usepackage{setspace}  %% Zur Setzung des Zeilenabstandes
\usepackage{babel}     %% Sprachen-Unterstuetzung
\usepackage{calc}      %% ermoeglicht Rechnen mit Laengen und Zaehlern
\usepackage[T1]{fontenc}       %% Unterstutzung von Umlauten etc.
\usepackage[latin1]{inputenc}  %% 
%% in aktuellem Linux & MacOS X wird standardmaessig UTF8 kodiert!
%\usepackage[utf8]{inputenc}    %% Wenn latin1 nicht geht ...

\usepackage{amsmath,amssymb} %% zusaetzliche Mathe-Symbole

\usepackage{lmodern} %% type1-taugliche CM-Schrift als Variante zur
                     %% "normalen" EC-Schrift
%% Paket fuer bibtex-Datenbanken
\usepackage[comma,numbers,sort&compress]{natbib}
%% modified by A.Quadt, 01.09.2010
% \bibliographystyle{plainnat}
\bibliographystyle{bthesis}
% added by A.Quadt, 01.09.2010
\usepackage{longtable}
%\usepackage[it, bf]{caption}
\usepackage{amsfonts}
\usepackage{amsmath}
\usepackage{mathrsfs}
\usepackage{epsfig}
%\usepackage[clearempty]{titlesec}
\usepackage{booktabs}
\usepackage{hhline}
\usepackage{array}
\usepackage{floatflt}
\usepackage{graphicx}
\usepackage{dcolumn}
\usepackage{bm}
\usepackage{mathrsfs} 
\usepackage{amssymb}
\usepackage{siunitx}

\usepackage{amsfonts}
\usepackage{amsmath}
\usepackage{mathrsfs}
\usepackage{xspace}
%

\sisetup{
  inter-unit-product 	=	$\cdot$,
  fraction-function   	= 	\nicefrac,
  load-configurations 	= 	abbreviations,
  per-mode            	= 	fraction,
  separate-uncertainty	=	true,
  output-decimal-marker	=	{.}
  }
\DeclareSIUnit\barn{b}
\usepackage{float}


\setlength{\oddsidemargin}{0cm}
\setlength{\evensidemargin}{0cm}
\setlength{\topmargin}{-1cm}
\setlength{\textheight}{23cm}
\setlength{\textwidth}{16cm}
\setlength{\parindent}{0cm}

\pagestyle{headings}

\renewcommand{\sectfont}{\bfseries\rmfamily}
\renewcommand{\floatpagefraction}{0.7}
\renewcommand{\textfraction}{0.1}

% Experiments
\newcommand{\dzero}      {D\O\xspace}
\newcommand{\cdf}        {CDF\xspace}
\newcommand{\uubar}      {\mbox{$u\bar{u}$}\xspace}
\newcommand{\ddbar}      {\mbox{$d\bar{d}$}\xspace}
\newcommand{\ccbar}      {\mbox{$c\bar{c}$}\xspace}
\newcommand{\ssbar}      {\mbox{$s\bar{s}$}\xspace}
\newcommand{\ttbar}      {\mbox{$t\bar{t}$}\xspace}
\newcommand{\bbbar}      {\mbox{$b\bar{b}$}\xspace}
\newcommand{\wjets}      {\mbox{$W + 4\; jets$}\xspace}
\newcommand{\pttbar}     {\mbox{$p_{t\bar{t}}$}\xspace}
\newcommand{\pwjets}     {\mbox{$p_{W +4 \; jets}$}\xspace}
\newcommand{\ljets}      {\mbox{$\ell$+jets}\xspace}
\newcommand{\ejets}      {\mbox{$e$+jets}\xspace}
\newcommand{\mujets}     {\mbox{$\mu$+jets}\xspace}

% Laboratories
\newcommand{\fermilab}  {{F{\textsc{ermilab}}}\xspace}
\newcommand{\tevatron}  {{T{\textsc{evatron}}}\xspace}
\newcommand{\opal}      {{O{\textsc{pal}}}\xspace}
\newcommand{\cern}      {{C{\textsc{ern}}}\xspace}
\newcommand{\fnal}      {{F{\textsc{nal}}}\xspace}
\newcommand{\atlas}     {{A{\textsc{tlas}}}\xspace}
\newcommand{\lhc}       {{L{\textsc{hc}}}\xspace}
\newcommand{\lhcb}      {{L{\textsc{hc}}}{\scriptsize{b}}\xspace}
\newcommand{\lep}       {{L{\textsc{ep}}}\xspace}
\newcommand{\slc}       {{S{\textsc{lc}}}\xspace}
\newcommand{\pep}       {{P{\textsc{ep}}}\xspace}
\newcommand{\petra}     {{P{\textsc{etra}}}\xspace}
\newcommand{\hera}      {{H{\textsc{era}}}\xspace}
\newcommand{\lepaleph}  {{A{\textsc{leph}}}\xspace}
\newcommand{\delphi}    {{D{\textsc{elphi}}}\xspace}
\newcommand{\leplthree} {{L{\textsc{3}}}\xspace}
\newcommand{\lepopal}   {{O{\textsc{pal}}}\xspace}
\newcommand{\doris}     {{D{\textsc{oris}}}\xspace}
\newcommand{\isr}       {{I{\textsc{sr}}}\xspace}
\newcommand{\desy}      {{D{\textsc{esy}}}\xspace}
\newcommand{\kek}       {{K{\textsc{ek}}}\xspace}
\newcommand{\slac}      {{S{\textsc{lac}}}\xspace}
\newcommand{\tristan}   {{T{\textsc{ristan}}}\xspace}
\newcommand{\cms}       {{C{\textsc{ms}}}\xspace}
\newcommand{\alice}     {{A{\textsc{lice}}}\xspace}
\newcommand{\zeus}      {{Z{\textsc{eus}}}\xspace}
\newcommand{\hone}      {{H{\textsc{1}}}\xspace}
\newcommand{\minuit}    {{M{\textsc{inuit}}}\xspace}
\newcommand{\herwig}    {{H\textsc{erwig}}\xspace}
\newcommand{\acermc}    {{A\textsc{cerMC}}\xspace}
\newcommand{\evtgen}    {{E\textsc{vtgen}}\xspace}
\newcommand{\mcfm}      {{M\textsc{cfm}}\xspace}
\newcommand{\mcatnlo}   {{M\textsc{c@nlo}}\xspace}
\newcommand{\sherpa}    {{S\textsc{herpa}}\xspace}
\newcommand{\jimmy}     {{J\textsc{immy}}\xspace}
\newcommand{\cteq}      {{C\textsc{teq}}\xspace}
\newcommand{\pythia}    {{P\textsc{ythia}}\xspace}
\newcommand{\jetnet}    {{J\textsc{etnet}}\xspace}
\newcommand{\isajet}    {{I\textsc{sajet}}\xspace}
\newcommand{\jetset}    {{J\textsc{etset}}\xspace}
\newcommand{\vecbos}    {{V\textsc{ecbos}}\xspace}
\newcommand{\alpgen}    {{A\textsc{lpgen}}\xspace}
\newcommand{\vegas}     {{V\textsc{egas}}\xspace}
\newcommand{\gnu}       {{G\textsc{nu}}\xspace}
\newcommand{\onetop}    {{O\textsc{neTop}}\xspace}
\newcommand{\ztop}      {{Z\textsc{Top}}\xspace}
\newcommand{\toprex}    {{T\textsc{opRex}}\xspace}
\newcommand{\singletop} {{S\textsc{ingleTop}}\xspace}
\newcommand{\madgraph}  {{M\textsc{adgraph}}\xspace}
\newcommand{\madevent}  {{M\textsc{adevent}}\xspace}
\newcommand{\comphep}   {{C\textsc{omphep}}\xspace}
\newcommand{\qq}        {{Q\textsc{q}}\xspace}
\newcommand{\tauola}    {{T\textsc{auola}}\xspace}
\newcommand{\geant}     {{G\textsc{eant}}\xspace}
\newcommand{\GEANT}     {{G\textsc{eant}}\xspace}
\newcommand{\amegic}    {{A\textsc{megic++}}\xspace}

\newcommand{\met}       {\mbox{$\not\!\!E_{\mathrm{T}}$}\xspace}
\newcommand{\metcal}    {\mbox{$\not\!\!E_{Tcal}$}\xspace}
\newcommand{\MET}       {$\not\!\!E_{\mathrm{T}}$}
\newcommand{\lowmet}    {low-\mbox{$\not\!\!E_{\mathrm{T}}$}-QCD\xspace}
\newcommand{\lumi}      {$\mathcal{L}$\xspace}
\newcommand{\intlumi}   {$\int\mathcal{L}\,\mathrm{d}t$\xspace}

\newcommand{\runi}      {Run~I\xspace}  %% For Tevatron! (Roman numerals)
\newcommand{\runii}     {Run~II\xspace}
\newcommand{\LHCruni}   {Run~1\xspace}  %% For LHC! (Arabic numerals)
\newcommand{\LHCrunii}  {Run~2\xspace}
\newcommand{\LHCruniii}  {Run~3\xspace}

\newcommand{\tabheadfont}[1]{\textbf{#1}} %% Tabellenkopf in Fett
\usepackage{booktabs}                      %% Befehle fuer besseres Tabellenlayout
\usepackage{longtable}                     %% umbrechbare Tabellen
\usepackage{array}                         %% zusaetzliche Spaltenoptionen

%% umfangreiche Pakete fuer Symbole wie \micro, \ohm, \degree, \celsius etc.
\usepackage{textcomp,gensymb}

%\usepackage{SIunits} %% Korrektes Setzen von Einheiten
%\usepackage{units}   %% Variante fuer Einheiten

%% Hyperlinks im Dokument; muss als eines der letzten Pakete geladen werden
\usepackage[pdfstartview=FitH,      % Oeffnen mit fit width
            breaklinks=true,        % Umbrueche in Links, nur bei pdflatex default
            bookmarksopen=true,     % aufgeklappte Bookmarks
            bookmarksnumbered=true  % Kapitelnummerierung in bookmarks
            ]{hyperref}

%% Weiter benoetigte Pakete: datenumber
%% Falls dieses Paket nicht in der Installation vorhanden ist,
%% kann es von der Seite mit diesem Template heruntergeladen werden
%% und in einem LaTeX bekanntem Verzeichnis installiert werden (notfalls
%% dem Verzeichnis mit der Arbeit).
\begin{document}
%%
%%                   Ab hier muessen die Anpassungen geschehen
%%
%% Hier den eigenen Namen einsetzen
\ThesisAuthor{Siemen}{Aulich}
%% Hier den Geburtsort einsetzen
\PlaceOfBirth{Emden}
%% Titel Arbeit. Das erste Argument ist der deutsche, das zweite der
%% englische Titel.
\ThesisTitle{}{Improving Event Reconstruction for Dileptonic Decays of Top Quarks Pairs Using Machine Learning}
%% Erst- und Zweitgutacher/in
%% Ist der/die Betreuer/in nicht identisch mit dem/r Erstgutachter/in,
%% muss diese/r als optionales Argument angegeben werden.
%% Diese Angaben beziehen sich auf Institut-externe BetreuerInnen und sollten nur in Ausnahmen relevant sein.
%%\FirstReferee[Dr.\ \ldots]{Prof.\ Dr.\ \dots} % fuer externe Betreung
\FirstReferee{Prof.~Dr.~Stan Lai}                % fuer interne Betreung
%% Optionen mit Stand 01. Januar 2014:
%% Prof.~Dr.~Ariane Frey
%% Priv.Doz.~Dr.~J{\"o}rn Gro{\ss}e-Knetter
%% Prof.~Dr.~Hans Hofs{\"a}ss
%% Prof.~Dr.~Arnulf Quadt
%% Jun.Prof.~ Steffen Schumann
\Institute{DESY Hamburg}
\SecondReferee{Dr.~Katharina Behr}
%% added by A.Quadt, 31.08.2010
%% Referenz Nummer der Bachelorarbeit im Institut
\CourseName{M.Phy.1610: Development and Realization of Scientific
Projects in Theoretical Physics}
%%
%% Beginn und Ende des Anfertigungszeitraumes
\ThesisBegin{6}{4}{2025}
\ThesisEnd{4}{12}{2026}
%% DO NOT TOUCH THESE LINES!!!!
\frontmatter
\maketitle
\cleardoublepage
\begin{abstract}
  Here the key results of the thesis can be presented in about
  half a page.
  \bigskip\par
  \textbf{Keywords:} Physics, Bachelor thesis
\end{abstract}

%% Ende des Vorspanns
\cleardoublepage
\mainmatter   %% Anfang Hauptteil

\chapter{Introduction}
\chapter{Introduction}
\label{chap:introduction}
The top quark, being the heaviest known elementary particle, plays a crucial role in the Standard Model of particle physics. Its unique properties and interactions make it an excellent probe for testing the limits of our current understanding of fundamental forces and searching for potential new physics beyond the Standard Model. It is one of the most studied processes at the Large Hadron Collider (\lhc) at \cern.
Top quarks are predominantly produced in pairs (\ttbar) via the strong interaction in proton-proton collisions at the \lhc. Each top quark decays almost exclusively into a W boson and a b-quark. The W boson can further decay either leptonically (into a charged lepton and a neutrino) or hadronically (into a pair of quarks). The dileptonic decay channel, where both W bosons decay leptonically, results in a final state with two charged leptons, two b-jets, and two neutrinos. This channel, while having a lower branching ratio compared to other decay modes, offers a cleaner experimental signature due to the presence of high-\pt leptons and reduced hadronic activity. However, the presence of two neutrinos in the final state poses significant challenges for event reconstruction, as they evade detection, leading to missing transverse energy (\met) in the event.

Due to its clean signature and sensitivity to various top quark properties, the dileptonic \ttbar channel is widely used for precision measurements of the top properties or spin correlations. Two recent highlights in top quark physics are the stringent tests of spin correlations and quantum effects in pairs of top quarks, and the observation of a possible quasi-bound state resonance in the $t\bar{t}$ invariant mass spectrum. Both effects are predominantly studied in the dilepton decay channel of the top quark in a mass range close to the $t\bar{t}$ production threshold. The latter presents a very exciting opportunity to extend our current understanding of non-relativistic Quantum Chromodynamics.

Both analyses rely on measurements of angular distributions and correlations between the decay products of the top quarks. Probing this system requires a precise reconstruction of the top quarks, which is complicated by the presence of the two neutrinos. While analytical neutrino regression strategies have been studied extensively, the problem of correctly assigning b-jets to their parent top quarks remains largely unstudied. In the all-hadronic and lepton+jets channels, various methods for jet assignment have been developed and employed successfully. However, in the dileptonic channel, the jet assignment problem was often simplified or overlooked due to a focus on neutrino reconstruction. However, many of the sensitive variables used in $t\bar{t}$ precision measurements depend critically on the correct assignment of the jets. Misassignments can lead to significant biases and degrade the resolution of reconstructed observables, ultimately limiting the precision of measurements.

Inspired by the success of machine learning architectures in tackling the assignment challenge for hadronic decay channels, this work investigates using a transformer model for the dilepton channel. It introduces the necessary theoretical background in \Cref{chap:background} before detailing the use of machine learning in particle physics in \Cref{chap:machine-learning}. The reconstruction methods are outlined in \Cref{chap:reconstruction-methods}, followed by the development of the machine learning-based jet assignment in \Cref{chap:ml-jet-selection}. For this, different transformer architectures are explored and optimized. The performance of the developed methods is evaluated in \Cref{chap:performance-evaluation}, where thee impact on relevant physics observables is assessed. Finally, \Cref{chap:conclusion} summarizes the findings and discusses potential future directions for further improving dileptonic \ttbar event reconstruction.

\chapter{Theoretical Background}
\chapter{Theoretical Background}
\label{chap:background}
\section{The Standard Model of Particle Physics}
\label{sec:standard_model}
\begin{figure}[h]
    \centering
    \includegraphics[width=0.8\textwidth]{figures/standard_model.pdf}
    \caption{The particles of the Standard Model. Particle porperties taken from Ref. \cite{PDG2024}.}
    \label{fig:standard_model_particles}
\end{figure}
The Standard Model of Particle Physics (SM) provides the theoretical foundation describing three of the four elementary forces as well the particles making up the known matter.\\
The Figure \ref{fig:standard_model_particles} shows a tabular overview of the known particle of the Standard Model.
The SM classifies the elementary particles into fermions and bosons.
Fermions are the building blocks of matter and have half-integer spin values. They are further divided into quarks and leptons.
Quarks experience all four fundamental forces, while leptons do not participate in the strong interaction.
Bosons are the force carriers of the fundamental interactions and have integer spin values.
The photon, W and Z bosons, and gluons mediate the electromagnetic, weak, and strong forces, respectively.
The Higgs boson is responsible for giving mass to other particles through the Higgs mechanism.\\
Based on the coupling of the weak interaction, two particles are grouped into one iso-spin doublet.
The iso-spin is the quantum number related to the weak interaction. Two quarks form one doublet and one lepton and its corresponding neutrino form another doublet.
Based on the different masses, two iso-spin doublets are grouped into what is called a generation.
There are three generations of fermions, with each generation containing heavier particles than the previous one.

\section{Top Quark Physics}
\label{sec:top_quark_physics}
The positive iso-spin partner of the third generation quark doublet is the top quark ($t$). With a mass of about $173\,\mathrm{GeV}/c^2$ \cite{PDG2024}, it is the heaviest known elementary particle.
Due to its high mass, the top quark has a very short lifetime of approximately $5\times10^{-25}\,\mathrm{s}$ \cite{PDG2024}, which is shorter than the timescale for hadronization.


\chapter{Methodology}
\chapter{Machine Learning in High Energy Physics}
Since this work focuses on improving the event reconstruction of dileptonic \ttbar decays using machine learning techniques, this chapter provides an overview of the fundamental concepts and methodologies employed in machine learning. 

\section{Supervised Learning}
Machine learning can be broadly categorized into supervised and unsupervised learning. In supervised learning, the model is trained on a labeled dataset, where each input data point is associated with a corresponding target output. The goal of the model is to learn a mapping from inputs to outputs, enabling it to make accurate predictions on unseen data.
\subsection{Monte Carlo Training Data}
In high energy physics, supervised learning is commonly performed by training mdoels on simulated datasets. The way the Monte Carlo Event modelling is performed in particle physics, makes it naturally suited for supervised learning tasks.\\
Events are usually simulated using a cascade of different programs, each simulating a different state of the event. First the hard scattering process is simulated using matrix element generators like \textsc{MadGraph} \cite{madevent_1} or \textsc{Powheg} \cite{powheg}. These programs calculate the probabilities of different particle interactions based on the underlying physics theories, such as the Standard Model. Using these probabilities, they generate events that represent the initial state of the particles after the collision. The possible kinematic phase space is sampled according to these probabilities, resulting in a set of particles with specific momenta and energies. This type of event variables is often referred to as \textit{parton-level}.\\
Next, the parton showering and hadronization processes are simulated using programs like \textsc{Pythia} \cite{pythia}. Parton showering models the emission of additional particles from the initial partons, while hadronization simulates the formation of hadrons from quarks and gluons. These processes are crucial for accurately modelling the final state particles observed in detectors.\\
Finally, detector simulation programs like \textsc{Geant4} \cite{geant4} are used to simulate the interaction of particles with the detector material, producing realistic detector responses. This includes simulating the energy deposits in calorimeters, hits in tracking detectors, and other relevant signals.\\
While the actual detector response is simulated using complex detector simulation software, for many machine learning applications, a simplified representation of the detector response is sufficient. This can involve smearing the particle momenta and energies according to the detector resolution, applying efficiency corrections, and simulating the effects of pile-up.\\
This is because, for the reconstruction of the physics objects, such as jets, leptons, and missing transverse energy, highly sophisticated algorithms are used that already take into account the detector effects (Note, that these algortihms may also be machine learning based,). Therefore, the input features for machine learning models can often be derived directly from these reconstructed objects, rather than relying on the raw detector signals. The reconstructed object event variables are called \textit{reco-level}.
\subsection{Training}
During the training phase, the model is presented with a set of input features derived from the reco-level event variables, along with their corresponding target outputs, which are typically derived from the parton-level event variables. The model learns to map the input features to the target outputs by minimizing a loss function that quantifies the difference between the predicted and true values. Common loss functions include mean squared error for regression tasks and cross-entropy \cite{cross-entropy} loss for classification tasks.\\

\chapter{Results}
\include{chapters/04-results}

\chapter{Conclusion}
\include{chapters/05-conclusion}

\appendix
\chapter{Additional Plots}



\cleardoublepage
%% Bibliographie. Das Argument muss der Name der BIBTeX-Datenbank stehen.
%% Ein Beispiel fuer eine solche Datenbank finden Sie in bthesis_datenbank.bib
\bibliography{bthesis_datenbank} 

\chapter*{Danksagung}
Dank\dots

%% Dieser Befehl MUSS am Ende stehen und erzeugt die Erklaerung ueber die
%% benutzten Mittel
\begin{otherlanguage}{ngerman}
\Declaration
\end{otherlanguage}
\end{document}
