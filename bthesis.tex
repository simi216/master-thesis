\NeedsTeXFormat{LaTeX2e}[2005/12/01]
%%    2009/03/12 v1.0 GAUBM Vorlage fuer Abschlussarbeiten Physik
%% Template fuer Bachelor- und Masterarbeiten
%% an der Fakultaet fuer Physik (c) Thomas Pruschke der GA Universitaet
%% Verbesserungsvorschlaege bitte an pruschke@theorie.physik.uni-goettingen.de
%%
%% Benoetigte Pakete: datenumber
%%

%%%%%%%%%%%%%%%%%%%%%%%%%%%%%%%%%%%%%%%%%%%%%%%%%%%%%%%%%%%%%%%%%%%%%%
%%%%%%%%%% Bitte vor dem Veraendern diese Datei umbenennen! %%%%%%%%%%
%%%%%%%%%%%%%%%%%%%%%%%%%%%%%%%%%%%%%%%%%%%%%%%%%%%%%%%%%%%%%%%%%%%%%%

%% scrbook - Ersatz fuerr LaTeX book Klasse aus dem KOMA Script
%% Moegliche Optionen: diejenigen der Klasse scrbook ausser titlepage
%% Updates, Fixes und Modifikationen von Boris Lemmer, 07.01.2015

%% deutsche Arbeit:
\documentclass[bachelor,       %% Typ der Arbeit: bachelor oder master
               twoside,        %% zweiseitiges Layout
               BCOR10mm,       %% Bindekorrektur 10 mm
%               liststotoc,nomtotoc,bibtotoc, %% Aufnahme der div. Verzeichnisse
                                              %% ins Inhaltsverzeichnis
%               english,ngerman, %% Alternativspr. Englisch, Dokumentspr. Deutsch
               ngerman,english  %% Alternativspr. Deutsch, Dokumentspr. Englisch
%               final,          %% Endversion; draft fuer schnelles Kompilieren
               ]{GAUBM}

\usepackage{setspace}  %% Zur Setzung des Zeilenabstandes
\usepackage{babel}     %% Sprachen-Unterstuetzung
\usepackage{calc}      %% ermoeglicht Rechnen mit Laengen und Zaehlern
\usepackage[T1]{fontenc}       %% Unterstutzung von Umlauten etc.
\usepackage[latin1]{inputenc}  %% 
%% in aktuellem Linux & MacOS X wird standardmaessig UTF8 kodiert!
%\usepackage[utf8]{inputenc}    %% Wenn latin1 nicht geht ...

\usepackage{amsmath,amssymb} %% zusaetzliche Mathe-Symbole

\usepackage{lmodern} %% type1-taugliche CM-Schrift als Variante zur
                     %% "normalen" EC-Schrift
%% Paket fuer bibtex-Datenbanken
\usepackage[comma,numbers,sort&compress]{natbib}
%% modified by A.Quadt, 01.09.2010
% \bibliographystyle{plainnat}
\bibliographystyle{bthesis}
% added by A.Quadt, 01.09.2010
\usepackage{longtable}
%\usepackage[it, bf]{caption}
\usepackage{amsfonts}
\usepackage{amsmath}
\usepackage{mathrsfs}
\usepackage{epsfig}
%\usepackage[clearempty]{titlesec}
\usepackage{booktabs}
\usepackage{hhline}
\usepackage{array}
\usepackage{floatflt}
\usepackage{graphicx}
\usepackage{dcolumn}
\usepackage{bm}
\usepackage{mathrsfs} 
\usepackage{amssymb}
\usepackage{siunitx}

\usepackage{amsfonts}
\usepackage{amsmath}
\usepackage{mathrsfs}
\usepackage{xspace}
%

\sisetup{
  inter-unit-product 	=	$\cdot$,
  fraction-function   	= 	\nicefrac,
  load-configurations 	= 	abbreviations,
  per-mode            	= 	fraction,
  separate-uncertainty	=	true,
  output-decimal-marker	=	{.}
  }
\DeclareSIUnit\barn{b}
\usepackage{float}


\setlength{\oddsidemargin}{0cm}
\setlength{\evensidemargin}{0cm}
\setlength{\topmargin}{-1cm}
\setlength{\textheight}{23cm}
\setlength{\textwidth}{16cm}
\setlength{\parindent}{0cm}

\pagestyle{headings}

\renewcommand{\sectfont}{\bfseries\rmfamily}
\renewcommand{\floatpagefraction}{0.7}
\renewcommand{\textfraction}{0.1}

% Experiments
\newcommand{\dzero}      {D\O\xspace}
\newcommand{\cdf}        {CDF\xspace}
\newcommand{\uubar}      {\mbox{$u\bar{u}$}\xspace}
\newcommand{\ddbar}      {\mbox{$d\bar{d}$}\xspace}
\newcommand{\ccbar}      {\mbox{$c\bar{c}$}\xspace}
\newcommand{\ssbar}      {\mbox{$s\bar{s}$}\xspace}
\newcommand{\ttbar}      {\mbox{$t\bar{t}$}\xspace}
\newcommand{\bbbar}      {\mbox{$b\bar{b}$}\xspace}
\newcommand{\wjets}      {\mbox{$W + 4\; jets$}\xspace}
\newcommand{\pttbar}     {\mbox{$p_{t\bar{t}}$}\xspace}
\newcommand{\pwjets}     {\mbox{$p_{W +4 \; jets}$}\xspace}
\newcommand{\ljets}      {\mbox{$\ell$+jets}\xspace}
\newcommand{\ejets}      {\mbox{$e$+jets}\xspace}
\newcommand{\mujets}     {\mbox{$\mu$+jets}\xspace}

% Laboratories
\newcommand{\fermilab}  {{F{\textsc{ermilab}}}\xspace}
\newcommand{\tevatron}  {{T{\textsc{evatron}}}\xspace}
\newcommand{\opal}      {{O{\textsc{pal}}}\xspace}
\newcommand{\cern}      {{C{\textsc{ern}}}\xspace}
\newcommand{\fnal}      {{F{\textsc{nal}}}\xspace}
\newcommand{\atlas}     {{A{\textsc{tlas}}}\xspace}
\newcommand{\lhc}       {{L{\textsc{hc}}}\xspace}
\newcommand{\lhcb}      {{L{\textsc{hc}}}{\scriptsize{b}}\xspace}
\newcommand{\lep}       {{L{\textsc{ep}}}\xspace}
\newcommand{\slc}       {{S{\textsc{lc}}}\xspace}
\newcommand{\pep}       {{P{\textsc{ep}}}\xspace}
\newcommand{\petra}     {{P{\textsc{etra}}}\xspace}
\newcommand{\hera}      {{H{\textsc{era}}}\xspace}
\newcommand{\lepaleph}  {{A{\textsc{leph}}}\xspace}
\newcommand{\delphi}    {{D{\textsc{elphi}}}\xspace}
\newcommand{\leplthree} {{L{\textsc{3}}}\xspace}
\newcommand{\lepopal}   {{O{\textsc{pal}}}\xspace}
\newcommand{\doris}     {{D{\textsc{oris}}}\xspace}
\newcommand{\isr}       {{I{\textsc{sr}}}\xspace}
\newcommand{\desy}      {{D{\textsc{esy}}}\xspace}
\newcommand{\kek}       {{K{\textsc{ek}}}\xspace}
\newcommand{\slac}      {{S{\textsc{lac}}}\xspace}
\newcommand{\tristan}   {{T{\textsc{ristan}}}\xspace}
\newcommand{\cms}       {{C{\textsc{ms}}}\xspace}
\newcommand{\alice}     {{A{\textsc{lice}}}\xspace}
\newcommand{\zeus}      {{Z{\textsc{eus}}}\xspace}
\newcommand{\hone}      {{H{\textsc{1}}}\xspace}
\newcommand{\minuit}    {{M{\textsc{inuit}}}\xspace}
\newcommand{\herwig}    {{H\textsc{erwig}}\xspace}
\newcommand{\acermc}    {{A\textsc{cerMC}}\xspace}
\newcommand{\evtgen}    {{E\textsc{vtgen}}\xspace}
\newcommand{\mcfm}      {{M\textsc{cfm}}\xspace}
\newcommand{\mcatnlo}   {{M\textsc{c@nlo}}\xspace}
\newcommand{\sherpa}    {{S\textsc{herpa}}\xspace}
\newcommand{\jimmy}     {{J\textsc{immy}}\xspace}
\newcommand{\cteq}      {{C\textsc{teq}}\xspace}
\newcommand{\pythia}    {{P\textsc{ythia}}\xspace}
\newcommand{\jetnet}    {{J\textsc{etnet}}\xspace}
\newcommand{\isajet}    {{I\textsc{sajet}}\xspace}
\newcommand{\jetset}    {{J\textsc{etset}}\xspace}
\newcommand{\vecbos}    {{V\textsc{ecbos}}\xspace}
\newcommand{\alpgen}    {{A\textsc{lpgen}}\xspace}
\newcommand{\vegas}     {{V\textsc{egas}}\xspace}
\newcommand{\gnu}       {{G\textsc{nu}}\xspace}
\newcommand{\onetop}    {{O\textsc{neTop}}\xspace}
\newcommand{\ztop}      {{Z\textsc{Top}}\xspace}
\newcommand{\toprex}    {{T\textsc{opRex}}\xspace}
\newcommand{\singletop} {{S\textsc{ingleTop}}\xspace}
\newcommand{\madgraph}  {{M\textsc{adgraph}}\xspace}
\newcommand{\madevent}  {{M\textsc{adevent}}\xspace}
\newcommand{\comphep}   {{C\textsc{omphep}}\xspace}
\newcommand{\qq}        {{Q\textsc{q}}\xspace}
\newcommand{\tauola}    {{T\textsc{auola}}\xspace}
\newcommand{\geant}     {{G\textsc{eant}}\xspace}
\newcommand{\GEANT}     {{G\textsc{eant}}\xspace}
\newcommand{\amegic}    {{A\textsc{megic++}}\xspace}

\newcommand{\met}       {\mbox{$\not\!\!E_{\mathrm{T}}$}\xspace}
\newcommand{\metcal}    {\mbox{$\not\!\!E_{Tcal}$}\xspace}
\newcommand{\MET}       {$\not\!\!E_{\mathrm{T}}$}
\newcommand{\lowmet}    {low-\mbox{$\not\!\!E_{\mathrm{T}}$}-QCD\xspace}
\newcommand{\lumi}      {$\mathcal{L}$\xspace}
\newcommand{\intlumi}   {$\int\mathcal{L}\,\mathrm{d}t$\xspace}

\newcommand{\runi}      {Run~I\xspace}  %% For Tevatron! (Roman numerals)
\newcommand{\runii}     {Run~II\xspace}
\newcommand{\LHCruni}   {Run~1\xspace}  %% For LHC! (Arabic numerals)
\newcommand{\LHCrunii}  {Run~2\xspace}
\newcommand{\LHCruniii}  {Run~3\xspace}

\newcommand{\tabheadfont}[1]{\textbf{#1}} %% Tabellenkopf in Fett
\usepackage{booktabs}                      %% Befehle fuer besseres Tabellenlayout
\usepackage{longtable}                     %% umbrechbare Tabellen
\usepackage{array}                         %% zusaetzliche Spaltenoptionen

%% umfangreiche Pakete fuer Symbole wie \micro, \ohm, \degree, \celsius etc.
\usepackage{textcomp,gensymb}

%\usepackage{SIunits} %% Korrektes Setzen von Einheiten
%\usepackage{units}   %% Variante fuer Einheiten

%% Hyperlinks im Dokument; muss als eines der letzten Pakete geladen werden
\usepackage[pdfstartview=FitH,      % Oeffnen mit fit width
            breaklinks=true,        % Umbrueche in Links, nur bei pdflatex default
            bookmarksopen=true,     % aufgeklappte Bookmarks
            bookmarksnumbered=true  % Kapitelnummerierung in bookmarks
            ]{hyperref}

%% Weiter benoetigte Pakete: datenumber
%% Falls dieses Paket nicht in der Installation vorhanden ist,
%% kann es von der Seite mit diesem Template heruntergeladen werden
%% und in einem LaTeX bekanntem Verzeichnis installiert werden (notfalls
%% dem Verzeichnis mit der Arbeit).
\begin{document}
%%
%%                   Ab hier muessen die Anpassungen geschehen
%%
%% Hier den eigenen Namen einsetzen
\ThesisAuthor{Niklas}{Gr\"un}
%% Hier den Geburtsort einsetzen
\PlaceOfBirth{Zeven}
%% Titel Arbeit. Das erste Argument ist der deutsche, das zweite der
%% englische Titel.
\ThesisTitle{Testmessung mit Modulen f\"ur den ATLAS ITk Pixel Detektor}{Test with Modules for the ATLAS ITk Pixel Detector}
%% Erst- und Zweitgutacher/in
%% Ist der/die Betreuer/in nicht identisch mit dem/r Erstgutachter/in,
%% muss diese/r als optionales Argument angegeben werden.
%% Diese Angaben beziehen sich auf Institut-externe BetreuerInnen und sollten nur in Ausnahmen relevant sein.
%%\FirstReferee[Dr.\ \ldots]{Prof.\ Dr.\ \dots} % fuer externe Betreung
\FirstReferee{Prof.~Dr.~Arnulf Quadt}                % fuer interne Betreung
%% Optionen mit Stand 01. Januar 2014:
%% Prof.~Dr.~Ariane Frey
%% Priv.Doz.~Dr.~J{\"o}rn Gro{\ss}e-Knetter
%% Prof.~Dr.~Hans Hofs{\"a}ss
%% Prof.~Dr.~Arnulf Quadt
%% Jun.Prof.~ Steffen Schumann
\Institute{II. Physikalischen Institut}
\SecondReferee{Priv.Doz.~Dr.~J{\"o}rn Gro{\ss}e-Knetter}
%% added by A.Quadt, 31.08.2010
%% Referenz Nummer der Bachelorarbeit im Institut
\ReferenceNumber{II.Physik-UniG{\"o}-BSc-2023/09}
%%
%% Beginn und Ende des Anfertigungszeitraumes
\ThesisBegin{23}{10}{2023}
\ThesisEnd{4}{2}{2024}
%% DO NOT TOUCH THESE LINES!!!!
\frontmatter
\maketitle
\cleardoublepage
%% Zusammenfassung. Falls nicht gewuenscht, bitte auskommentieren.
\begin{otherlanguage}{english} % Boris Lemmer, 07.01.2015
\begin{abstract}
  Hier werden auf einer halben Seite die Kernaussagen der Arbeit
  zusammengefasst.
%% Optional: Stichwoerter. Wenn nicht gewuenscht, koennen die beiden
%% folgenden Zeilen geloescht werden
  \bigskip\par
  \textbf{Stichw{\"o}rter:} Physik, Bachelorarbeit
\end{abstract}
\end{otherlanguage}
%% So laesst sich in die andere Sprache umschalten (Englisch bzw. Deutsch)
\begin{otherlanguage}{english} % Boris Lemmer, 07.01.2015
\begin{abstract}
  Here the key results of the thesis can be presented in about
  half a page.
  \bigskip\par
  \textbf{Keywords:} Physics, Bachelor thesis
\end{abstract}
\end{otherlanguage}

%% Ende des Vorspanns
\cleardoublepage
%% Ab hier 1 1/2 facher Zeilenabstand (durch setspace-Paket)
\onehalfspacing
%% Erzeugt Inhaltsverzeichnis
\tableofcontents

%% Hier kann man seine Bezeichnungsweisen erklaeren. Falls nicht
%% benoetigt, bis einschliesslich \end{nomenclature} auskommentieren
\begin{nomenclature}
%% Fuer die Berechnung der Spaltenbreiten muss \usepackage{calc}
%% geladen sein!
\section*{Lateinische Buchstaben}
\noindent
\begin{longtable}[l]{p{0.2\textwidth}p{0.7\textwidth-6\tabcolsep}p{0.1\textwidth}}
  \tabheadfont{Variable}&\tabheadfont{Bedeutung}&\tabheadfont{Einheit}\\\midrule\endhead
  $A$ & Querschnittsfl{\"a}che & $\unit{m^2}$\\
 % $c$ & Geschwindigkeit & $\unitfrac{m}{s}$
\end{longtable}
\section*{Griechische Buchstaben}
\begin{longtable}[l]{p{0.2\textwidth}p{0.7\textwidth-6\tabcolsep}p{0.1\textwidth}}
  \tabheadfont{Variable}&\tabheadfont{Bedeutung}&\tabheadfont{Einheit}\\\midrule\endhead
  $\alpha$  & Winkel & $\unit{\degree}$; --\\
%  $\varrho$ & Dichte & $\unitfrac{kg}{m^3}$
\end{longtable}
\section*{Indizes}
\begin{longtable}[l]{p{0.2\textwidth}p{0.8\textwidth-4\tabcolsep}}
  \tabheadfont{Index}&\tabheadfont{Bedeutung}\\\midrule\endhead
  m & Meridian\\
  $r$ & Radial
\end{longtable}
\section*{Abk{\"u}rzungen}
\begin{longtable}[l]{p{0.2\textwidth}p{0.8\textwidth-4\tabcolsep}}
  \tabheadfont{Abk"urzung}&\tabheadfont{Bedeutung}\\\midrule\endhead
  2D & zweidimensional\\
  3D & dreidimensional\\
  max & maximal
\end{longtable}
\end{nomenclature}
%% \listoftables und \listoffigures sollten nur bei genuegender Anzahl Tabellen
%% verwendet werden
%\listoffigures
%\listoftables

\mainmatter   %% Anfang Hauptteil

\chapter{Einleitung}
Diese Vorlage \verb!GAUBM!
f{\"u}r Bachelor- bzw.\ Masterarbeiten ist eine {\"U}berarbeitung der
Vorlage von Simon Dreher f{\"u}r Abschlu{\ss}arbeiten am
Institut f{\"u}r Mikrosystemtechnologie (IMTEK)
an der Universit{\"a}t Freiburg. Die eigentliche Datei mit der Klassendefinition
ist \verb!GAUBN.cls!, die Sie zusammen mit dieser Datei erhalten haben. Weitere
Dateien sind \verb!datenumber.sty! und die zugeh{\"o}rigen Sprachdefinitionen
\verb!\datenumber*.ldf!. Im Verzeichnis \verb!figures! finden sich die
von der Klasse ben{\"o}tigten Logos (Universit{\"a}t und Physik) sowie Beispielbilder
f{\"u}r die {\"U}bersetzung dieser Beispieldatei (\verb!bthesis.tex!).
Sie k{\"o}nnen diese Datei als Vorlage f{\"u}r Ihre Arbeit nutzen und entsprechend
modifizieren. Bitte denken Sie daran, sie vorher unter einem eigenen Namen
abzuspeichern.
Um die Datei anzupassen, gehen Sie wie folgt vor:

Bei den Parametern zu \verb!\documentclass[...]{GAUBM}! in der Pr{\"a}ambel
kann man durch Umschalten
zwischen \verb!english,ngerman! und \verb!ngerman,english! eine
deutsche Arbeit (erste Variante) mit Englisch als Alternativsprache bzw.\ eine
englische Arbeit (zweite Variante) mit deutsch als Alternativsprache
waehlen. Im laufenden Text kann man mit 
\begin{verbatim}
\begin{otherlanguage}{english/ngerman}
...
\end{otherlanguage}
\end{verbatim}
zur alternativen Sprache wechseln.

Nach \verb!\begin{document}! m{\"u}ssen zuerst ein paar Befehle mit
Information {\"u}ber die Arbeit aufgerufen werden:
\begin{enumerate}
\item \verb!\ThesisAuthor{Vorname}{Nachname}!: Die Argumente sind der
Vorname und Nachname der Autorin bzw.\ des Autors der Arbeit.
\item \verb!\PlaceOfBirth{Wohnort}!: Der Geburtsort der Autorin bzw.\ des Autors.
\item \verb!\ThesisTitle{Deutscher Titel}{English title}!: Der deutsche und englische Titel der Arbeit gem{\"a}{\ss} Antrag.
\item \verb!\Institute{Institut}!: Das Institut, an dem die Arbeit angefertigt wurde.
\item \verb!\FirstReferee[Betreuer/in]{Erste/r Gutachter/in}!: Voller Titel
und Name des/r Erstgutachter/in. Ist der Betreuer der Arbeit \emph{nicht}
identisch mit dem/r Erstgutachter/in, so mu{\ss} der volle Titel und der Name
des/r Betreuer/in als optionales Argument in eckigen Klammern erscheinen.
\item \verb!\SecondReferee{Zweite/r Gutachter/in}!: Voller Titel
und Name des/r Zweitgutachter/in.
\item \verb!\ThesisBegin{Tag}{Monat}{Jahr}!: Datum des Beginns der Anfertigung
der Arbeit gem{\"a}{\ss} Antrag.
\item \verb!\ThesisEnd{Tag}{Monat}{Jahr}!: Datum der Fertigstellung der Arbeit.
\item Optional kann mit
\begin{verbatim} 
\begin{abstract}
...
\end{abstract}
\end{verbatim}
eine maximal eine halbe Seite lange Zusammenfassung eingef{\"u}gt werden.

Falls man die Zusammenfassung in der alternativen Sprache verfassen m{\"o}chte,
dann geht das mit der Befehlsfolge
\begin{verbatim} 
\begin{otherlanguage}{english/ngerman}
\begin{abstract}
...
\end{abstract}
\end{otherlanguage}
\end{verbatim}
\end{enumerate}

\chapter{Grundlagen}
In diesem Kapitel werden die theoretischen Grundlagen erl{\"a}utert.

Wichtige Gleichungen, die in der Arbeit h{\"a}ufiger zitiert werden,
sollten eine Gleichungsnummer erhalten.
\begin{equation}
  \label{eq:pythagoras}
  a^2+b^2=c^2
\end{equation}
Zum Beispiel wird in Gleichung~\ref{eq:pythagoras} der Satz des Pythagoras
angegeben.

Gerade im Bereich der Grundlagen wird viel Literatur zitiert, z.B.\
\cite{Menz97}. Falls
mehrere Literaturzitate auf einmal zitiert werden, ist folgendes
z.B.\ m{\"o}glich \cite{Horn90,DINEN6232,Menz97,Knuth84}.

\section{Unterkapitel Gliederungsebene 2}
Hier sollte etwas Text stehen.
\subsection{Unterkapitel Gliederungsebene 3}
Noch ein paar Beispiele zu Abbildungen und Tabellen:

Abbildung~\ref{fig:bildplatzhalter} verdeutlicht \dots

Wie die Abb.~\ref{fig:bildplatzhalter} und
Tab.~\ref{tab:tabellenplatzhalter} verdeutlichen \dots

\begin{figure}
  \centering
  \includegraphics[width=0.5\linewidth]{figures/bild}
  \caption{Bildbeschreibung}
  \label{fig:bildplatzhalter}
\end{figure}

Text\dots
\begin{table}[ht]
  \centering
  \caption{Tabellenbeschreibung, im Gegensatz zu Bildbeschreibungen (z.B. Abb.~\ref{fig:bildplatzhalter}) sollte diese immer \emph{oberhalb} der Tabelle gesetzt sein!}
  \label{tab:tabellenplatzhalter}
  \begin{tabular}{llll}
    \toprule
    $A$-Wert&$B$-Wert&$C$-Wert&$D$-Wert\\
    \midrule
    aaaaaa&bbbbbbb&cccccc&ddddddd\\
    aaaaaa&bbbbbbb&cccccc&ddddddd\\
    \bottomrule
  \end{tabular}
\end{table}

Text\dots

\chapter{Experimentelle Vorgehensweise}
Text\dots

\chapter{Hinweise zum Formulieren und Zitieren}

%
% introduction
%
These comments result from reading and refereeing quite a number of
Bachelor theses and are meant to help mastering the \LaTeX{} technology
and some formalities.

%
% choice of language
%
Choose the language of your document well. Writing it in German has
the advantage that it is most likely your mother tongue and hence you
will find it easy to express yourself clearly and present arguments
precisely. The disadvantage is that most other scientists, for example
of your ATLAS collaborators, will not be able to read it. Writing it
in English you will potentially have worldwide readers. On the other
hand, the quality of your document might not reflect the quality of
your scientific work and hence you might get lower grades than
possible. We can and will help, but it is your choice and your
responsibility.

%
% British English
%
Typically, scientific documents are written in English. In Europe, and
in particular in the ATLAS collaboration, we use British English (BE),
not American English (AE). That means, for example:
\begin{itemize}
\item colour, not color
\item behaviour, not behavior
\item flavour, not flavor
\item Aluminium, not Aluminum
\item etc.
\end{itemize}

Names of chemical elements are typically written as starting with a
capital letter, i.e. Gold, Argon, Aluminium, etc. Also foreign words
stay in the original style of spelling, for example German words like
Bremsstrahlung, Ansatz, Gedankenexperiment etc. start with a capital
letter.

When refering to leptons in formulae, we typically use the symbol
$\ell$, which you can obtain in \LaTeX{} as $\$${$\backslash$ell}$\$$.

In colloquial English, you can contract the words ``does not'' to
``doesn't'', in written English you cannot do that.

Usage of indefinite articles ``a'' and ``an'': In general, you use
``an'' if the subsequent word is pronounced as if it started with a
vowel (regardless how it is written), otherwise you use ``a''. Examples are:
\begin{enumerate}
\item[\textbf{ right}:] a pizza, an hour, an FPGA, an 8, a unit, a window
\item[\textbf{ wrong}:] an pizza, a hour, a FPGA, a 8, an unit, an window
\end{enumerate}

In physics texts, numbers and units occur frequently. Units are
generally written in Roman fonts, while the numbers or formulae
automatically come out in italics in the \LaTeX{} math mode. You can
enforce the Roman fonts for units by adding ``$\backslash$mathrm$\{\}$''
around them. Also, there should always be a little space between the
number and the unit, which you can enforce by adding ``$\backslash $,''
in between. Examples are:
\begin{enumerate}
\item[\textbf{ right}:] $\tau \approx 5 \times 10^{-25}\,\mathrm{s}$ from
                  $\$\backslash$tau $\backslash$approx 5 $\backslash$times 10$\hat{\ }$\{-25\}$\backslash$, $\backslash$mathrm\{s\}$\$$
\item[\textbf{ }      ] $\mathcal{L} = 10^{34}\,\mathrm{cm}^{-2}\,\mathrm{s}^{-1}$ from
                  $\$\backslash$mathcal$\{$L$\} = 10\hat{\ }\{34\}{\backslash}$,$\backslash$mathrm$\{$cm$\}\hat{\ }\{-2\}\backslash$,$\backslash$mathrm$\{$s$\}\hat{\ }\{-1\}\$$
\item[\textbf{ wrong}:] $\tau \approx 5 \times 10^{-25}s$\; from
                  $\$\backslash$tau $\backslash$approx 5 $\backslash$times 10$\hat{\ }$\{-25\} s$\$$
\item[\textbf{ }      ] ${\cal{L}} = 10^{34}cm^{-2}s^{-1}$ from
                  $\$\{\backslash$cal$\{$L$\}\} = 10\hat{\ }\{34\}\textrm{ cm}\hat{\ }\{-2\}$s$\hat{\ }\{-1\}\$$
\end{enumerate}
%
As an alternative, you can use the package \texttt{siunitx} which comes with commands to typeset units correctly. Using the above examples, the \LaTeX{} code shortens to:
\begin{enumerate}
\item[\textbf{ right}:] $\tau \approx \SI{5e-25}{\s}$ from $\$\backslash$tau $\backslash$approx $\backslash$SI$\{$5e-25$\}\{\backslash$s$\}\$$
\item[\textbf{ }      ] $\mathcal{L} = \SI{e34}{\cm\tothe{-2}\s\tothe{-1}}$ from $\$\backslash$mathcal$\{$L$\}$ = $\backslash$SI$\{$e34$\}\{\backslash$cm$\backslash$tothe$\{$-2$\}$ $\backslash$s$\backslash$tothe$\{$-1$\}\}\$$
\end{enumerate}

Make sure you have chosen the correct language setting in your \LaTeX{}
file as this defines, amongst other aspects, the hyphenation rules.

%
% apostrophes
%
The German language does essentially not know the usage of
apostrophes. When writing your thesis in German, forget about
apostrophes and do not use them. If you write it in English, use
apostrophes correctly. There are two situations where apostrophes can
be used in English:
\begin{itemize}
\item In colloquial English you can contract words, for example ``is
  not'' becomes ``isn't'', ``does not'' becomes ``doesn't'', and
  ``Peter is hungry'' becomes ``Peter's hungry''. As mentioned before,
  this construction does exist in English, but it is not to be used in
  formal English such as a thesis.
\item Native German speakers are very familiar with the case
  ``Genitiv'', for example: ``Das Antwortverhalten des ATLAS
  Kalorimeters auf Hadronen ...''. The correct English translation is
  ``The ATLAS calorimeters' response to hadrons ...'', i.e. the
  apostrophe is appended at the end of the word. You cannot say ``The
  ATLAS calorimeter's response to hadrons ...''. That would be
  Gerglish or Denglish and is not accepted as a thesis language.
\end{itemize}

 
%
% spell checker
%
When writing a document, it is really helpful to use a spell check.
You can spell check in the standard editor Emacs as long as Ispell is
installed on the computer you are using. Type \texttt{ which ispell} to
figure out if it is installed. If not, please contact your local
system administrator. If it is, to check an open document, enter the
following command (``M-x'' stands for the simultaneous pressing of the
keys ``AltG'' and ''x''): \texttt{ M-x ispell-[option]}

Replace [option] with either \texttt{ buffer}, \texttt{ region}, \texttt{ string},
or \texttt{ word}. (To check the entire document in the current buffer,
use the buffer option.)

Once the spell checker starts, Emacs will highlight each word it does
not recognize, and prompt you for an action. Emacs will try to find
correct spellings for the current word in the dictionaries at its
disposal. If it finds any, it will list them in a separate buffer at
the top of the screen. Each suggested spelling will be preceded by a
character in parentheses. To change the word to one of these
spellings, simply type that character. Other commands are summarized
below: 

\begin{table}[h]
\begin{tabular}{lp{12cm}}
Key         & Action\\ \hline
r           & Enter a new spelling by hand \\
Spacebar    & Leave the word unchanged\\
a           & Accept this spelling for all buffers during the current
              editing session only\\
i           & Accept this spelling from now on, adding it to your personal
              dictionary in your home directory\\
q           & Quit the spell checker\\
X (Shift-x) & Halt spell checking at current location so that
              later it will restart there.\\ \hline
\end{tabular}
\end{table}

You man find more information about \texttt{ Ispell} by typing \texttt{ man
  ispell} or by searching on the web.

Another spell checking option is to use the Spell functions. You
invoke them similarly to the Ispell functions, in that you must enter:
\texttt{ M-x spell-[option]}

Replace [option] with either \texttt{ buffer}, \texttt{ region}, \texttt{ string},
or \texttt{ word}.  Unlike the Ispell functions, these functions will not
provide you with a list of suggested spellings; for all words not
found in Emacs' dictionary, you must type in the correct spelling by
hand. These functions are not available in some versions of Emacs.

When using \texttt{ xemacs} instead of \texttt{ emacs}, you can click on ``Cmds'',
``Spell-Check'', ``Check Buffer'' to start the spell checker.

%
% citations
%
Writing a Bachelor thesis is probably the first time you need to cite
other peoples' documents and publications. Here a few hints how to do
it:
\begin{itemize}
\item Do not reference \textit{ standard} pictures such as aerial views of
  CERN, ATLAS detector pictures, diagrams of quarks and leptons etc.
\item Do reference documents without detailed page number specification.
\item Try as much as possible to cite original papers, for example for
  the Standard Model, the Higgs mechanism or QCD, rather than random
  review articles, text books or even other students' theses. \texttt{
    Spires} at http://www.slac.stanford.edu/spires is really helpful
  for that.
\item Web pages cannot be used as references. They are not constant in
  time. The information you want to refer to might no longer be
  available on that page in the future. As a consequence, only cite
  time-stable documents, i.e. papers.
\item Citations belong to the sentence in which you want to make a
  statement based on some information found elsewhere. Hence, the
  reference $\left[ \dots \right]$ is part of the sentence, followed
  by the full
  stop. Do not write a full stop followed by the reference.\\
  \textbf{ Yes:} last words of the sentence \cite{bauer_orr_01}.\\
  \textbf{ No:} last words of the sentence. \cite{abe_prl74_1995}
\item Include a space between the last word before the reference and
  the square brackets of the reference, i.e. blabla
  \cite{abachi_prl74_1995,pdg2002} and not
  blabla\cite{abachi_prl74_1995,pdg2004}.
  When the reference(s) is/are subject of the sentence, you should add
  ``Ref.'' or ``Refs.'' before the square brackets, for example:
  ``As shown in Refs.~\cite{abachi_prl74_1995,pdg2002}'' and not just
  write ``As shown in~\cite{abachi_prl74_1995,pdg2002}''.
\item Use \texttt{bibtex} to deal with the references. An example bibtex
  file is included here. Out of the bibtex file, which can be view as
  a large database of possible papers and documents to cite, bibtex
  projects out only the ones that you do actually cite and
  automatically puts them in the right order. Also bibtex tried to
  make sure they are listed in the same format, i.e. author are all
  listed as first name plus surname or all as initial plus surname
  etc. Of course, this can only work if the required information is
  included in your bibtex file. In our style, we quote initials plus
  surnames.
\item For papers with several authors, typically only the first one is
  mentioned followed by an \textit{ et al.}. That is Latin and stands for
  ``et alii'' or ``et aliae'', depending on the gender people. It
  means ``and others'' and can be obtained in \LaTeX{} via
  $\backslash$etal  .
\end{itemize}

When refering to figures, tables, equations, sections, or chapters,
start the word in capital letters, add a tilde in order to inject a
space and then add the reference to the object, for example:
\begin{itemize}
  \item see Figure~\ref{fig:bildplatzhalter} from ``see
    Figure$\tilde{\ }\backslash$ref\{fig:bildplatzhalter\}''.
  \item see Table~\ref{tab:tabellenplatzhalter} from ``see
    Table$\tilde{\ }\backslash$ref\{tab:tabellenplatzhalter\}''.
  \item see Chapter ..., see Section ..., see Equation ...
\end{itemize}

In cases you work on data analysis, please remember that:
\begin{itemize}
\item no, the rapidity is {\underline{not}} Lorentz invariant!
\item PDF uncertainties cannot be estimated be calculating a cross
  section, for example, with two different PDF parametrisations and
  taking the difference of the two results. For that purpose, modern
  parton distributions functions come with error bands, CTEQ6 for
  example with 20 for positive and 20 for negative deviations.
\end{itemize}

If you have a look at the style file \texttt{ style.tex}, you will find a
number of common and frequently used abbreviations, such as \dzero,
\cdf, \ttbar, \ljets, \mujets, \tevatron, \cern, \pythia, \herwig, \geant, \met
etc. Try them out. Be aware that you should use \runii for \tevatron, but \LHCrunii for the \lhc (Roman vs. Arabic numerals).


\chapter{Ergebnisse}
Text\dots
\chapter{Diskussion}
Text\dots
\section{Unterkapitel}
Text\dots
\subsection{Unterkapitel}
Text\dots
\chapter{Zusammenfassung}
Text\dots

\appendix
\chapter{erster Anhang}
Text\dots
\chapter{zweiter Anhang}
Text\dots

\cleardoublepage
%% Bibliographie. Das Argument muss der Name der BIBTeX-Datenbank stehen.
%% Ein Beispiel fuer eine solche Datenbank finden Sie in bthesis_datenbank.bib
\bibliography{bthesis_datenbank} 

\chapter*{Danksagung}
Dank\dots

%% Dieser Befehl MUSS am Ende stehen und erzeugt die Erklaerung ueber die
%% benutzten Mittel
\begin{otherlanguage}{ngerman}
\Declaration
\end{otherlanguage}
\end{document}
