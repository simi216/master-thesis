\chapter{Introduction}
\label{chap:introduction}
The top quark, being the heaviest known elementary particle, plays a crucial role in the Standard Model of particle physics. Due to its high mass, it stands out in two ways; for one, as having the highest Yukawa-coupling and secondly, as being the shortest lived elementary particle. While, the Yukawa-coupling makes it very interesting for Higgs physics, the short lifetime of the top quark presents it as the only window allowing us to look at the bare coupling of quarks. The top quark decays almost exclusively into a $W$ boson and a $b$-quark. These properties make processes involving top quarks as some of the most studied at the Large Hadron Collider (\lhc) at \cern.

Top quarks are predominantly produced in pairs (\ttbar) via the strong interaction in proton-proton collisions at the \lhc. Each $W$ boson coming from the top decay can decay either leptonically (into a charged lepton and neutrino) or hadronically (into a pair of quarks).
The dileptonic decay channel, where both $W$ bosons decay leptonically, results in a final state with two charged leptons, two $b$-jets, and two neutrinos. This channel, while having a lower branching ratio compared to other decay modes, offers a cleaner experimental signature due to the presence of leptons and reduced hadronic activity. Leptons in particular can be reconstructed with efficiency in a hadron collider environment.
However, the presence of two neutrinos in the final state poses significant challenges for event reconstruction, as they evade detection, leading to missing transverse energy (\met) in the event.

Due to its clean signature and sensitivity to various top-quark properties, the dileptonic \ttbar channel is widely used for precision measurements of the top properties and spin correlations in particular. Two recent highlights in top-quark physics are the tests of spin correlations and quantum entanglement in pairs of top quarks~\cite{ATLAS2024Entanglement}, and the observation of a possible quasi-bound state resonance in the $t\bar{t}$ invariant mass spectrum~\cite{ATLAS2025ThresholdEnhancement,CMS2025Pseudoscalar}. Both effects are predominantly studied in the dilepton decay channel of the top quark in a mass range close to the on-shell production threshold of two top quarks $m(t\bar{t})\gtrsim 2\cdot m_t\approx 350\,\unit{\giga\electronvolt}$. Compared to hadronic decays, the leptons allow a more accurate probing of the top quarks spin by preserving a larger fraction of the spin information~\cite{spin_analyzing_power,spin_analyzing_power_jets}.

Both analyses rely on measurements of angular distributions and correlations between the decay products of the top quarks. Probing this system requires a precise reconstruction of the top quarks from their decay products, which is complicated by the presence of the two neutrinos. While analytical neutrino regression strategies have been studied extensively, the problem of correctly assigning $b$-jets to their parent top quarks remains largely unstudied. In channels with hadronically decaying top quarks, various methods for jet assignment have been developed and employed successfully~\cite{spanet}. However, in the dileptonic channel, the jet assignment problem was often simplified or overlooked due to a focus on neutrino reconstruction. However, many of the sensitive variables used in measurements of spin correlations depend critically on the correct assignment of the jets. Misassignments degrade the resolution of reconstructed observables, ultimately limiting the precision of measurements.

Inspired by the success of machine-learning architectures in tackling the assignment challenge for hadronic decay channels, this work investigates the use of  a transformer model for the dilepton channel. It introduces the necessary theoretical background in \Cref{chap:background} before detailing the use of machine learning in particle physics in \Cref{chap:machine-learning}. The reconstruction methods are outlined in \Cref{chap:reconstruction-methods}, followed by the development of the machine learning-based jet assignment in \Cref{chap:ml-jet-selection}. For this, different transformer architectures are explored and optimized. The performance of the developed methods is evaluated in \Cref{chap:performance-evaluation}, where the impact on relevant physics observables is assessed. Finally, \Cref{chap:conclusion} summarizes the findings and discusses potential future directions for further improving dileptonic \ttbar event reconstruction.