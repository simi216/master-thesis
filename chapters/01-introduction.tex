\chapter{Introduction}
\label{chap:introduction}
The top quark, being the heaviest known elementary particle, plays a crucial role in the Standard Model of particle physics. Its unique properties and interactions make it an excellent probe for testing the limits of our current understanding of fundamental forces and searching for potential new physics beyond the Standard Model. It is one of the most studied processes at the Large Hadron Collider (\lhc) at \cern.
Top quarks are predominantly produced in pairs (\ttbar) via the strong interaction in proton-proton collisions at the \lhc. Each top quark decays almost exclusively into a W boson and a b-quark. The W boson can further decay either leptonically (into a charged lepton and a neutrino) or hadronically (into a pair of quarks). The dileptonic decay channel, where both W bosons decay leptonically, results in a final state with two charged leptons, two b-jets, and two neutrinos. This channel, while having a lower branching ratio compared to other decay modes, offers a cleaner experimental signature due to the presence of high-\pt leptons and reduced hadronic activity. However, the presence of two neutrinos in the final state poses significant challenges for event reconstruction, as they evade detection, leading to missing transverse energy (\met) in the event.

Due to its clean signature and sensitivity to various top quark properties, the dileptonic \ttbar channel is widely used for precision measurements of the top properties or spin correlations. Two recent highlights in top quark physics are the stringent tests of spin correlations and quantum effects in pairs of top quarks, and the observation of a possible quasi-bound state resonance in the $t\bar{t}$ invariant mass spectrum. Both effects are predominantly studied in the dilepton decay channel of the top quark in a mass range close to the $t\bar{t}$ production threshold. The latter presents a very exciting opportunity to extend our current understanding of non-relativistic Quantum Chromodynamics.

Both analyses rely on measurements of angular distributions and correlations between the decay products of the top quarks. Probing this system requires a precise reconstruction of the top quarks, which is complicated by the presence of the two neutrinos. While analytical neutrino regression strategies have been studied extensively, the problem of correctly assigning b-jets to their parent top quarks remains largely unstudied. In the all-hadronic and lepton+jets channels, various methods for jet assignment have been developed and employed successfully. However, in the dileptonic channel, the jet assignment problem was often simplified or overlooked due to a focus on neutrino reconstruction. However, many of the sensitive variables used in $t\bar{t}$ precision measurements depend critically on the correct assignment of the jets. Misassignments can lead to significant biases and degrade the resolution of reconstructed observables, ultimately limiting the precision of measurements.

Inspired by the success of machine learning architectures in tackling the assignment challenge for hadronic decay channels, this work investigates using a transformer model for the dilepton channel. It introduces the necessary theoretical background in \Cref{chap:background} before detailing the use of machine learning in particle physics in \Cref{chap:machine-learning}. The reconstruction methods are outlined in \Cref{chap:reconstruction-methods}, followed by the development of the machine learning-based jet assignment in \Cref{chap:ml-jet-selection}. For this, different transformer architectures are explored and optimized. The performance of the developed methods is evaluated in \Cref{chap:performance-evaluation}, where thee impact on relevant physics observables is assessed. Finally, \Cref{chap:conclusion} summarizes the findings and discusses potential future directions for further improving dileptonic \ttbar event reconstruction.