\chapter{Analysis and Reconstruction}
\label{chap:reconstruction-methods}
\section{Data and Simulation}
\label{sec:data-and-simulation}

This section describes the data and simulation samples used in this analysis. At the current stage of this thesis, only simulation samples have been studied; the data samples will be described in the final version. The simulated samples are produced using Monte Carlo (MC) event generators, which model the physics processes occurring in proton-proton collisions at the LHC.

While a complete analysis requires detailed investigation of background processes contributing to the signal regions, this thesis focuses on developing and evaluating reconstruction methods. Therefore, only the signal process of \ttbar production with dileptonic decays is described here. The background processes and their simulation will be discussed in the final thesis.

\subsection{Simulation of Dileptonic Top Quark Pair Events}

The signal process of \ttbar production with dileptonic decays is simulated using \powheg~\cite{powheg} at next-to-leading order (NLO) in perturbative quantum chromodynamics (QCD). The \powheg generator is interfaced with \pythia~\cite{pythia} for parton showering, hadronization, and underlying event modelling.

To link the reco-level particles to the parton-level objects, a parton matching is performed. The parton-level objects are defined as the top quarks and their decay products after final state radiation before hadronization. The matching is done by iteratively associating reconstructed jets to partons based on angular distance criteria. The decay products of the top quarks are matched within a cone of $\Delta R \leq 0.3$ and $\Delta R\leq 0.1$ for jets and leptons ($e$,$\mu$) respectively. If multiple reconstructed objects are found within this cone, the one with the smallest $\Delta R$ is chosen.


\subsection{Object Definition}

The reconstruction of physics objects—electrons, muons, jets, and missing transverse energy (\met)—is performed using the standard \atlas reconstruction algorithms~\cite{atlas_tdr}.

\paragraph{Electrons} are required to have transverse momentum \pt $> 5~\unit{\giga\electronvolt}$ and pseudorapidity $|\eta| < 2.47$, excluding the transition region between the barrel and endcap calorimeters ($1.37 < |\eta| < 1.52$). Electrons must satisfy the Tight identification criteria and be isolated from other activity in the detector.

\paragraph{Muons} are reconstructed using information from both the inner detector and the muon spectrometer. They are required to have \pt $> 5~\unit{\giga\electronvolt}$ and $|\eta| < 2.5$. Muons must satisfy the Tight identification criteria and be isolated.

\paragraph{Jets} are reconstructed using the anti-$k_t$ algorithm \cite{anti_k_t_jet_2} with a radius parameter of $R = 0.4$. They are required to have \pt $> 25~\unit{\giga\electronvolt}$ and $|\eta| < 2.5$. B-tagging is performed using the GN2 algorithm~\cite{gn2_btagger} to identify jets originating from b-quarks.

\paragraph{Missing Transverse Energy} (\met) is calculated from the negative vector sum of the transverse momenta of all reconstructed objects.

\subsection{Event Selection}
Events are selected to contain exactly two oppositely charged leptons (electrons or muons). One of the leptons must have \pt $> 25~\unit{\giga\electronvolt}$, while the other must have \pt $> 5~\unit{\giga\electronvolt}$. At least two jets are required, with at least one being b-tagged at the $77\%$ working point (WP). Additional selection criteria are applied to suppress background contributions, which will be detailed in the final thesis.

For the training and evaluation of the reconstruction methods, only events that pass the selection criteria and have a successful parton matching are considered.

\section{Event Reconstruction}
\label{sec:reconstruction-methods}
Reconstructing the kinematics of dileptonic \ttbar events requires an assignment of reconstructed objects to the parton-level decay products. This section describes the different reconstruction methods developed and evaluated in this thesis.

While the leptons are directly identified from the reconstructed objects, the assignment of jets to the b-quarks from the top quark decays is ambiguous due to the presence of multiple jets in the event. Additionally, the two neutrinos from the W boson decays are not directly detected, leading to missing information in the event reconstruction.
Based on this, the reconstruction breaks down into two main tasks:
\paragraph{Neutrino Regression} Estimating the momenta of the two neutrinos using the measured \met and other event kinematics.
\paragraph{Jet Assignment} Assigning the reconstructed jets to the b-quarks from the top quark decays.
\subsection{Neutrino Regression Methods}
\label{sec:neutrino_regression}
Due to the current stage of the thesis, only one baseline method for neutrino regression has been used. The method of choice for the final thesis will be determined after evaluating multiple approaches.

The baseline method used here is called $\nu^2$-Flows \cite{nu_square_flows} and uses normalizing flows to model the conditional probability distribution of the neutrino momenta given the observed event kinematics. The model is trained on simulated dileptonic \ttbar events, where the true neutrino momenta are known from the parton-level information. The trained model can then be used to sample possible neutrino momenta for new events, allowing for a probabilistic reconstruction of the event kinematics. For each event, multiple samples of neutrino momenta are drawn from the model, and the one with the highest probability density is selected as the reconstructed neutrino momenta.
A detailed outline of the $\nu^2$-Flows method and its implementation can be found in Ref.~\cite{nu_square_flows}. For this thesis, this method serves to establish a baseline for neutrino regression to evaluate the impact of improved jet assignment methods. Note, however, that this method has not been used in \atlas analyses yet. However, due to matters of software compatibility, it was the only viable option at this stage of the thesis.
Since, $\nu^2$-Flows processes the full event information to make a prediciton for the neutrino momenta, it is agnostic to the jet assignment. This is drastically different from the conventional methods, which were used in previous \atlas analyses \cite{entanglement_top_quarks}. Which are interdependent with the jet assignment. These are mostly based on exploiting the kinematic constraints of the event, such as the invariant masses of the W bosons and top quarks outlined in \Cref{sec:kinematic_constraints}.
\subsection{Jet Assignment Methods}
\label{sec:jet_assignment}
This thesis explores several methods for jet assignment, ranging from traditional algorithms to machine learning approaches. From recent publications e.g. \cite{entanglement_top_quarks}, two conventional methods have been implemented as baselines:

\paragraph{Jet-Selection}
Both methods start by selecting the two $b$-jet candidates in the event based on the $b$-tagging at the $77\%$ WP. If more than two $b$-tagged jets are present, the two with the highest \pt are chosen. If only one $b$-tagged jet is found, the non-tagged jet with the highest $b$-tagging score is selected as the second candidate.
\begin{figure}[H]
    \centering
    \begin{minipage}[t]{0.48\textwidth}
        \textbf{$\chi^2$-Method}\\[0.5em]
        For both possible assignments of the two b-jet candidates to the parton-level $b$-quarks, the invariant masses of the reconstructed top quarks are computed
        \begin{equation}
            \label{eq:chi_square}
            \chi^2 = (m_{b_1,{\nu},\ell^+} - m_{t})^2+(m_{b_2,\bar{\nu},\ell^-} - m_{t})^2
        \end{equation}
        where $m_t = 172.5~\unit{\giga\electronvolt}$ is the top quark mass. The assignment with the smaller $\chi^2$ value is chosen.
    \end{minipage}
    \hfill
    \begin{minipage}[t]{0.48\textwidth}
        \textbf{$\Delta R$-Method}\\[0.5em]
        This method assigns the $b$-jet candidates based on their angular distance to the leptons. 
        \begin{equation}
            \Delta R = \sqrt{(\Delta\phi)^2 + (\Delta\eta)^2}
        \end{equation}
        The jet lepton pair with the smaller $\Delta R$ value is assigned to each other. The remaining jet is assigned to the other lepton.
    \end{minipage}
\end{figure}
Note, that the $\chi^2$-Method relies on the reconstructed neutrino momenta to compute the invariant masses of the top quarks. Therefore, it is inherently linked to the neutrino regression method used. In contrast, the $\Delta R$-Method is independent of the neutrino momenta.
Further, it should be noted, that due to its reliance on the neutrino reconstruction and the use of $\nu^2$-Flows here, the $\chi^2$-method should also be considered a maching learning based approach.

The $\Delta R$-Method was used in previous \atlas analyses \cite{entanglement_top_quarks} for events, that failed the kinematic reconstruction required for the $\chi^2$-Method. However, in this thesis, both methods are evaluated on the full dataset for comparison.