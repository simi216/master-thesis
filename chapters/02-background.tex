\chapter{Theoretical Background}
\label{chap:background}
\section{The Standard Model of Particle Physics}
\label{sec:standard_model}
\begin{figure}[h]
    \centering
    \includegraphics[width=0.8\textwidth]{figures/standard_model.pdf}
    \caption{The particles of the Standard Model. Particle porperties taken from Ref. \cite{PDG2024}.}
    \label{fig:standard_model_particles}
\end{figure}
The Standard Model of Particle Physics (SM) provides the theoretical foundation describing three of the four elementary forces as well the particles making up the known matter.\\
The Figure \ref{fig:standard_model_particles} shows a tabular overview of the known particle of the Standard Model.
The SM classifies the elementary particles into fermions and bosons.
Fermions are the building blocks of matter and have half-integer spin values. They are further divided into quarks and leptons.
Quarks experience all four fundamental forces, while leptons do not participate in the strong interaction.
Bosons are the force carriers of the fundamental interactions and have integer spin values.
The photon, W and Z bosons, and gluons mediate the electromagnetic, weak, and strong forces, respectively.
The Higgs boson is responsible for giving mass to other particles through the Higgs mechanism.\\
Based on the coupling of the weak interaction, two particles are grouped into one iso-spin doublet.
The iso-spin is the quantum number related to the weak interaction. Two quarks form one doublet and one lepton and its corresponding neutrino form another doublet.
Based on the different masses, two iso-spin doublets are grouped into what is called a generation.
There are three generations of fermions, with each generation containing heavier particles than the previous one.

\section{Top Quark Physics}
\label{sec:top_quark_physics}
The positive iso-spin partner of the third generation quark doublet is the top quark ($t$). With a mass of about $173\,\mathrm{GeV}/c^2$ \cite{PDG2024}, it is the heaviest known elementary particle.
Due to its high mass, the top quark has a very short lifetime of approximately $5\times10^{-25}\,\mathrm{s}$ \cite{PDG2024}, which is shorter than the timescale for hadronization.
