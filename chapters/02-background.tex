\chapter{Theoretical Background}
\label{chap:background}
\section{The Standard Model of Particle Physics}
\label{sec:standard_model}

\begin{figure}[ht]
    \centering
    \includegraphics[width=0.8\textwidth]{figures/standard_model.pdf}
    \caption{The particles of the Standard Model. Particle properties taken from Ref.~\cite{PDG2024}.}
    \label{fig:standard_model_particles}
\end{figure}
The Standard Model (SM) of particle physics is a quantum field theory describing
three of the four fundamental interactions of nature—electromagnetic, weak and strong—and the elementary particles on which they act.
It is formulated as a renormalisable, Lorentz-invariant gauge theory with gauge group
\[
SU(3)_C \times SU(2)_L \times U(1)_Y,
\]
and its structure has been established in a series of seminal works~\cite{standardmodel0,standardmodel1,standardmodel2,standardmodel7,standardmodel5,standardmodel9}. 
A summary of the particle content is shown in \Cref{fig:standard_model_particles}.

The elementary particles fall into fermions and bosons. Fermions are spin-$\tfrac{1}{2}$ fields that constitute matter, organised into three generations of quarks and leptons. 
Quarks carry colour charge and participate in the strong interaction; leptons do not.
Bosons are integer-spin gauge fields mediating the fundamental forces: the photon for electromagnetism,
the $W^\pm$ and $Z$ bosons for the weak interaction,
and eight gluons for the strong interaction.
The Higgs boson completes the SM and is responsible for electroweak symmetry breaking (EWSB), giving mass to the weak gauge bosons and, through Yukawa couplings, to the fermions.
The SM Lagrangian can be written schematically as
\[
\mathcal{L}_\mathrm{SM} = \mathcal{L}_\mathrm{gauge}
+ \mathcal{L}_\mathrm{fermion}
+ \mathcal{L}_\mathrm{Yukawa}
+ \mathcal{L}_\mathrm{Higgs},
\]
where the first term contains the gauge kinetic terms and self-interactions, the second describes the fermions and gauge covariant derivatives,
the third encodes the Yukawa couplings to the Higgs field, and the last term contains the Higgs potential and EWSB mechanism.

Below, the three interaction sectors are briefly summarised, with emphasis on the aspects relevant for top-quark physics and LHC phenomenology.

\subsection{Electromagnetic Interaction}

The electromagnetic interaction is described by quantum electrodynamics (QED), the $U(1)_{\mathrm{EM}}$ gauge theory that remains after EWSB.
Its massless gauge boson, the photon, couples to electric charge and does not self-interact.
Because QED is an Abelian theory, its coupling runs only mildly with energy and remains perturbative at all experimentally accessible scales.

\subsection{Weak Interaction}

The weak interaction forms, together with electromagnetism, the electroweak theory based on the gauge group $SU(2)_L \times U(1)_Y$.
The vacuum expectation value of the Higgs field spontaneously breaks the symmetry to $U(1)_{\mathrm{EM}}$, giving masses to the weak gauge bosons through the Higgs mechanism~\cite{higgs1,higgs2,higgs3}. 
The resulting massive $W^\pm$ and $Z$ bosons mediate a short-range force.

A defining property of the weak interaction is its \emph{chiral} structure: 
charged-current interactions couple only to left-handed fermions and right-handed antifermions. 
This feature, predicted in~\cite{Parity_Violation_Theory} and confirmed in the Wu experiment~\cite{WU_Exp}, 
leads to maximal parity violation and characteristic angular distributions of weak decay products. The vertex factor of the charged-current interaction is given by 
\begin{equation}
    \frac{-ig}{2\sqrt{2}} \gamma^\mu (1 - \gamma^5),
\end{equation}
where $g$ is the weak coupling constant, $\gamma^\mu$ are the Dirac matrices. The projection operator $(1 - \gamma^5)$ leads to a coupling exclusively to the left-handed components.
Based on this, the left-handed particles are arranged in $SU(2)_L$ doublets corresponding, while the right-handed particles are singlets.

For quarks, the weak eigenstates $d',s',b'$ are not identical to their mass eigenstates $d,s,b$. This misalignment is described by the Cabibbo-Kobayashi-Maskawa (CKM) matrix~\cite{cabibbo,kobayashi_maskawa}, which encodes flavour-changing transitions in charged-current interactions.



\subsection{Strong Interaction}

The strong interaction is described by quantum chromodynamics (QCD), a non-Abelian gauge theory with gauge group $SU(3)_C$.
Quarks carry colour charge, while gluons carry a colour and anticolour index, giving rise to self-interactions through three- and four-gluon vertices.
These features lead to two fundamental properties:

\paragraph{Asymptotic freedom.}
The QCD coupling decreases at high momentum transfer~\cite{standardmodel6,standardmodel7}, allowing perturbative calculations of hard processes such as top-quark pair production.
In the high-energy regime of the LHC, quarks and gluons behave as effectively quasi-free particles.

\paragraph{Confinement.}
At low energies the QCD coupling becomes large, preventing coloured states from existing in isolation.
Instead, they form colour-neutral bound states—hadrons.
The transition from short-distance partons to long-distance hadrons involves nonperturbative dynamics encapsulated through hadronisation models.

A crucial consequence for collider physics is that quarks and gluons produced in hard interactions undergo a cascade of QCD emissions before hadronising into jets.
This process is typically described in terms of:

\begin{itemize}
\item \textbf{Parton distribution functions (PDFs)}, characterising the proton structure and entering via the QCD factorisation theorem~\cite{collins_soper_sterman}.
\item \textbf{Initial- and final-state radiation (ISR/FSR)}, arising from soft and collinear gluon emission.
\item \textbf{Parton showers}, which resum logarithmically enhanced emissions and are implemented in Monte Carlo generators such as PYTHIA,
first developed in~\cite{SJOSTRAND1985321,parton_shower} and extended in modern formulations~\cite{pythia}.
\item \textbf{Hadronisation models}, such as the Lund string model~\cite{lund_model,SJOSTRAND1984469}, which describe the confinement-driven formation of hadrons.
\end{itemize}

These aspects of QCD are essential for the reconstruction of hadronic final states at the LHC and for modelling top-quark production and decay.

\section{Top Quark Physics}
\label{sec:top-quark}
\begin{figure}[t]
    \centering
    \begin{subfigure}[b]{0.3\textwidth}
        \centering
        \includegraphics[width=\textwidth]{figures/feynman-diagrams/ttbar_production_qq_s_channel.pdf}
        \caption{}
        \label{fig:top_production_qq_s_channel}
    \end{subfigure}
    
    \begin{subfigure}[c]{0.3\textwidth}
        \centering
        \includegraphics[width=\textwidth]{figures/feynman-diagrams/ttbar_production_gg_t_channel.pdf}
        \caption{}
        \label{fig:top_production_gg_t_channel}
    \end{subfigure}
    \hfill
    \begin{subfigure}[c]{0.3\textwidth}
        \centering
        \includegraphics[width=\textwidth]{figures/feynman-diagrams/ttbar_production_gg_s_channel.pdf}
        \caption{}
        \label{fig:top_production_gg_s_channel}
    \end{subfigure}
    \hfill
    \begin{subfigure}[c]{0.3\textwidth}
        \centering
        \includegraphics[width=\textwidth]{figures/feynman-diagrams/ttbar_production_gg_u_channel.pdf}
        \caption{}
        \label{fig:top_production_gg_u_channel}
    \end{subfigure}
    \caption{Leading order Feynman diagrams for top quark pair production in \subrefb{fig:top_production_qq_s_channel} quark-antiquark annihilation and \subrefb{fig:top_production_gg_t_channel}, \subrefb{fig:top_production_gg_s_channel} \& \subrefb{fig:top_production_gg_u_channel} gluon fusion (ggF).}
    \label{fig:top_production}
\end{figure}
The top quark is the heaviest known elementary particle. Its predicted existence arose from the already observed CP-Violiation in the kaon system, which required a third generation of quarks to be explained within the SM framework~\cite{kobayashi_maskawa}. Furthermore, the discovery of the bottom quark in 1977~\cite{bottom_discovery} implied the existence of its weak isospin partner, the top quark. The top quark was eventually discovered in 1995 by the \cdf and \dzero collaborations at the \tevatron collider~\cite{top_discovery_cdf,top_discovery_d0}.
Currently, the world average of the top quark mass is measured to be~\cite{PDG2024}
\begin{equation}
    m_t = 172.56 \pm 0.31\,\unit{\giga\electronvolt}.
\end{equation}
Along with the highest mass, the top quark also has the strongest Yukawa coupling to the Higgs boson, making it interesting for studies of the Higgs mechanism and electroweak symmetry breaking.

Due to its large mass, the top quark also has the shortest lifetime of all quarks, about $5 \times 10^{-25}~\unit{\second}$~\cite{PDG2024}, which is shorter than the typical hadronisation timescale of $3 \times 10^{-24}~\unit{\second}$~\cite{hadronization_time_2}. Therefore, the top is the only quark that decays before it can hadronise, allowing for direct studies of its properties through its decay products.

Notably, the top quark also decays before its spin can decorrelate, preserving spin information in the angular distributions of its decay products~\cite{spin_decorrelation_time}. This unique feature enables precise measurements of top quark spin correlations and polarisation.
At hadron colliders like the \lhc, top quarks are predominantly produced in pairs via the strong interaction.

The leading order (LO) Feynman diagrams for top quark pair production are shown in \Cref{fig:top_production}. 
At the \lhc, the dominant production mechanism is gluon fusion (ggF).
This is due to the high gluon density in the proton at the relevant Bjorken-$x$ values for top quark production.

The CKM matrix element $V_{tb}$ is close to unity~\cite{PDG2024}, leading to a nearly exclusive decay of the top quark into a $W$ boson and a $b$ quark.
The $W$ can subsequently decay either leptonically into a charged lepton and a neutrino, or hadronically into a pair of quarks, with branching ratios of about $\text{BR}(W\to\ell\nu)=1/3$ and $\text{BR}(W\to q\bar{q})=2/3$, respectively \cite{PDG2024}.

Based on these $W$ decay modes, top quark pair events are categorised into three channels. In the \textbf{dileptonic channel}, both $W$ bosons decay leptonically ($W^+ \to \ell^+ \nu_\ell$, $W^- \to \ell^- \bar{\nu}_\ell$), resulting in two charged leptons, two neutrinos and two $b$ quarks in the final state. This channel has a branching ratio of approximately $1/9$. The \textbf{semileptonic channel} occurs when one $W$ boson decays leptonically and the other hadronically ($W \to q\bar{q}'$), yielding one charged lepton, one neutrino, four quarks (including two $b$ quarks) in the final state. This is the most common channel with a branching ratio of about $4/9$. Finally, in the \textbf{all-hadronic channel}, both $W$ bosons decay hadronically, producing six quarks (including two $b$ quarks) in the final state. This channel has the highest branching ratio of approximately $4/9$ but suffers from large multijet background contamination, especially in the busy environment of hadron colliders.

\subsection{Dileptonic Decay Channel}
\label{sec:top_decay_modes}
\begin{figure}[ht]
    \centering
    \includegraphics[width=0.5\textwidth]{figures/feynman-diagrams/g_to_dilepton.pdf}
    \caption{Leading order Feynman diagram for top quark pair production with dileptonic decay.}
    \label{fig:dilepton_ttbar_decay}
\end{figure}
The dileptonic decay channel, where both $W$ bosons decay leptonically, has a branching ratio of about $1/9$ and results in a final state with two charged leptons, two neutrinos and two $b$ quarks, as illustrated in \Cref{fig:dilepton_ttbar_decay}. Due to the lower branching ratio, the dileptonic channel has a smaller event yield compared to the other channels. However, it offers a cleaner signature with reduced background contamination, making it particularly suitable for precision measurements of top quark properties. The presence of two neutrinos in the final state leads to missing transverse energy (\met) in the event, complicating the full reconstruction of the top quark kinematics. Advanced techniques, such as kinematic fitting and multivariate analyses, are often employed to address these challenges and extract maximum information from the dileptonic events.
\subsubsection{Kinematic Constraints}
\label{sec:kinematic_constraints}
Assuming both $W$ bosons and top quarks are on-shell, the following kinematic constraints can be applied to the dileptonic decay channel
\begin{align}
    (p_{\ell_1} + p_{\nu_1})^2 &= m_W^2, \label{eq:w_boson1_mass}\\
    (p_{\ell_2} + p_{\nu_2})^2 &= m_W^2, \label{eq:w_boson2_mass}\\
    (p_{b_1} + p_{\ell_1} + p_{\nu_1})^2 &= m_t^2, \label{eq:top_quark1_mass}\\
    (p_{b_2} + p_{\ell_2} + p_{\nu_2})^2 &= m_t^2, \label{eq:top_quark2_mass}\\
    (\vec{p}_{\nu_1} + \vec{p}_{\nu_2})_{x} &= \met_x, \label{eq:met_x}\\
    (\vec{p}_{\nu_1} + \vec{p}_{\nu_2})_{y} &= \met_y. \label{eq:met_y}
\end{align}
If one assumes, the neutrinos to be massless, these six equations provide constraints on the six unknown components of the two neutrino momenta. Therefore, additional assumptions or techniques are required to fully reconstruct the event kinematics in the dileptonic channel. Due to the algebraic nature of the constraints, multiple solutions can exist for a given event, leading to ambiguities in the reconstruction \cite{sonnenschein_method, ellipse_method}. Various methods have been developed to address these challenges, including likelihood-based approaches \cite{neutrino_weighting_method}, matrix element methods, and machine learning techniques \cite{nu_square_flows}, which aim to optimally utilise the available information and improve the accuracy of the top quark kinematic reconstruction in dileptonic events.


\subsection{Spin Correlations in Top Quark Pairs}
\label{sec:ttbar_entanglement}
As before elaborated, the top quark decays before it can hadronise or its spin decorrelates. Therefore, the spin information of the top quark is directly transferred to its decay products. In top quark pair production, the spins of the top and antitop quarks are correlated due to the production mechanism. These spin correlations manifest in the angular distributions of the decay products, providing a unique opportunity to study the spin dynamics of the top quark.

Recent studies at the \lhc began to explore quantum entanglement in top quark pairs~\cite{ATLAS2024Entanglement}. Entanglement generally describes a quantum mechanical phenomenon where the quantum states of two or more particles become correlated in such a way that the state of one particle cannot be described independently of the state of the other(s), even when the particles are separated by large distances~\cite{schrödinger_entanglement}. This property is expressed as the system having a non-seperable density matrix. While a general description of this phenomenon is beyond the scope of this thesis, entanglement can be tested using specific criterions. One such criterion is the \emph{Peres-Horodecki criterion}~\cite{peres_criterion,entanglement_two_qbits}, which states that if the partial transpose of the density matrix of a bipartite system has at least one negative eigenvalue, then the system is entangled. Considering top quark pairs as a bipartite system of two spin-$\tfrac{1}{2}$ particles (qubits), this criterion can be applied to test for entanglement in their spin states.

In Ref.~\cite{entanglement_top_quarks} it was proposed to measure entanglement in top quark pairs produced at the \lhc by analysing the angular distributions of their decay products, specifically the charged leptons in the dileptonic decay channel. The study showed that by measuring the opening angle between the two leptons, one can construct an observable sensitive to the entanglement of the top quark spins. Due to the kinematics of the top quark pair production, the entanglement is expected to be more pronounced in certain regions of phase space, particularly at high invariant masses of the top quark pair system. One particular region of interest is the so-called \emph{threshold region}, where the top quark pair is produced with low relative velocity. In this region, the top quark is produced as a spin singlet state in about $\sim 80\%$ of the cases, leading to a strong entanglement signature \cite{toponium_2}. The \atlas collaboration has recently reported the first experimental evidence of quantum correlations consistent with entanglement in top quark pairs \cite{ATLAS2024Entanglement}.

\subsection{Top Quark Pair Bound State Effects}
\label{sec:ttbar_toponium}
Even though the top quark decays before it can hadronise, near the production threshold of top quark pairs, the strong interaction between the top and antitop quarks can lead to the formation of transient bound states, coined as toponium~\cite{toponium_1,toponium_2,toponium_3}. These bound state effects can influence the production cross section and kinematic distributions of top quark pairs near threshold. The \cms collaboration has recently observed an excess in the invariant mass distribution of top quark pairs near threshold, consistent with the presence of toponium bound state effects~\cite{CMS2025Pseudoscalar}. The Modelling of these effects is based on non-relativistic QCD (NRQCD) calculations, which account for the strong interaction dynamics between the top and antitop quarks in the near-threshold region~\cite{NRQCD_toponium_1, NRQCD_toponium_2}. Understanding and accurately modelling these bound state effects is crucial for precision measurements of top quark properties and for searches for new physics in top quark pair production.

This study is also investigating the threshold region of top quark pair production and makes use of angular correlations between the decay products as a probe of the underlying dynamics, including potential bound state effects.

\subsection{Spin Sensitive Angular Variables}
\label{sec:spin_sensitive_variables}
To study spin correlations and entanglement in top quark pairs, a pair of spin-sensitive angular variables are used.

To define the variables, the momenta of the leptons are first boosted into their respective parent top quark rest frames $\hat \ell_{t}^+$ and $\hat \ell^-_t$. The inner product of these unit vectors then defines the variable
\begin{equation}
    \label{eq:c_hel}
    c_{\text{hel}} = \hat \ell_{t}^+ \cdot \hat \ell_{t}^-.
\end{equation}
Analagously, one can define the variable $c_{\text{han}}$, where either of the lepton momenta components parallel to the parent top quark momentum in the \ttbar rest frame is flipped before taking the inner product \cite{bernreuther_nlospin_2004,bernreuther_spin_observables}.