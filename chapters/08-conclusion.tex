\chapter{Conclusion}
\label{chap:conclusion}
\section{Summary}
In this work, a novel machine learning-based approach for the assignment of $b$-jets to their parent top quarks in dileptonic $t\bar{t}$ events has been presented. By leveraging transformer architectures, significant improvements over traditional methods have been achieved, enhancing the accuracy of event reconstruction in this challenging decay channel.
Two primary model architectures were explored: one exploiting maximum symmetry between the input particles, and another incorporating physics-inspired features. Both models demonstrated superior performance compared to existing techniques, with the physics-inspired model showing a slight edge in assignment accuracy.
The models were trained and evaluated on simulated datasets, with careful consideration of hyperparameter tuning and training strategies.
Extensive performance evaluation and comparison with baseline methods confirmed the effectiveness of the proposed approach. The transformer models could improve the selection efficiency of correct $b$-jets and significantly enhance the assignment accuracy.
It was shown, that the improved jet assignment leads to better reconstruction of key physics observables, such as the $t\bar{t}$ invariant mass and angular distributions, which are crucial for precision measurements and searches for phenomena beyond the SM.

The dependence of the model performance on the physics regions was also investigated, revealing that the models' performance significantly drops in phase space regions with low $m(t\bar{t})$. While the models still outperform traditional methods in these regions, this observation highlights the need for further research to enhance model robustness across all kinematic regimes. Especially, since these regions are of particular interest for current measurements.

Based on these findings, the influence of the underlying physics on the model's performance was studied. For this, events were categorized based on various kinematic variables, and the accuracy of the methods was analysed within these categories. This analysis provided insights into the strengths and limitations of the model, guiding future improvements. The importance of the input features was also assessed using permutation importance techniques, revealing why the model struggles in certain kinematic regions.

To address the performance loss, it was investigated how emphasizing these regions during training can improve the performance of the model. By adjusting the loss function to give more weight to events in these challenging regions, a noticeable improvement in assignment accuracy was achieved, demonstrating the potential of targeted training strategies.

Overall, this work establishes a strong foundation for the application of advanced machine learning techniques in the reconstruction of dileptonic $t\bar{t}$ events, paving the way for future developments in this area.

\section{Outlook}
Building on the promising results of this study, several avenues for future research and development can be pursued to further enhance the performance and applicability of machine learning models in dileptonic $t\bar{t}$ event reconstruction.

For once, the integration of the jet assignment model with neutrino momentum regression techniques will be explored to create a comprehensive event reconstruction pipeline. This promises to enable a more holistic approach to reconstructing dileptonic $t\bar{t}$ events, potentially leading to even greater improvements in accuracy and resolution of physics observables.
Apart from that, the shared information processing for both processes can lead to an increased efficiency and reduced computational requirements. While the increase in performance was one of the major benefits of using machine learning in the all-hadronic channel, the machine learning based approach for the dilepton channel is computationally much more expansive than traditional methods. A combined model might help to mitigate this issue.

Additionally, further refinement of the model architectures and training strategies will be investigated. This includes exploring more sophisticated transformer variants, experimenting with different loss functions, and incorporating additional physics-inspired features to enhance model performance.

Lastly, the application of the developed models to real experimental data from the LHC could be pursued. This would involve validating the models' performance in a real-world setting and assessing their impact on precision measurements and new physics searches in dileptonic $t\bar{t}$ events.